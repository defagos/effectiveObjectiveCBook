42. [UIScreen bounds] is the complete screen (320 x 480 on iPhone)
       [UIScreen applicationFrame] is the complete screen, but without the status bar at the top. This is the area in which you will draw your widgets.
       
54b. Excellent concept: Etant donn� que le GUI iPhone doit tourner dans un thread (car pas thread-safe), appliquer le principe expliqu� ci-dessous:
�sidesteping the problem of threading bugs in your code. An iPhone app is one big event loop �  your classes have methods that the event loop calls in response to stuff happening on the device and in your app. When you use the URL loading system�s asynchronous APIs, the iPhone uses a different thread than the one running your app�s event loop to load the contents of the URL, and it makes callbacks via your apps event loop when data has been downloaded.�
En d�autres termes, poster les r�sultats d�un traitement asynchrone directement comme un event sur la main event loop. Ainsi, c�est le thread principal qui sera notifi� du changement.
En r�gle g�n�rale: Pour tout thread faisant tourner sa runloop, on peut appliquer l�approche suivante pour r�cup�rer du contenu de mani�re asynchrone: Une requ�te est effectu�e, avec delegate et m�thode � appeler sur ce delegate (en principe conform�ment � un protocole). La r�cup�ration du contenu est lanc�e de mani�re asynchrone et, d�s que le contenu a �t� r�cup�r�, le r�sultat est post� en tant qu�event de la runloop ayant initi� la requ�te. La seule contrainte est donc que le thread initiant la requ�te ait sa runloop. Voir �galement la documentation des m�thodes initWithRequest de NSURLConnection.h

Attention! Le concept de run loop est un des plus importants � assimiler!! Voir documentation Apple � ce sujet. C�est �galement elle qui re�oit les notifications!