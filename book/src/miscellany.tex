7. Always use localized strings, never hardcode strings in your code. This will make translation of your application easier

10. Localize the scope of your variables as much as possible: Unlike C (and like C++), Objective-C allows you to define a variable at the point it is used for the first time, it does not require you to declare all variables at the beginning of a block.

24. Comment your code as you write it. Commenting can be tedious for some people, especially if made all at once. By documenting and coding at the same time, your comments will be more accurate and (probably) less painful to write

32. Static NSString objects are created using @�My little NSString�, C-string objects with �My little C-string�. Do not mix both, and always use NSString in Objective-C code.

51. �Javadoc� du code: Voir NSURLConnection.h

62. When writing libraries, write them in such a way they can be used
indifferently with garbage collection or reference counting

64. When polishing classes (but not too early in the class' life, otherwise
this work might be lost because of refactorings), do not forget:
 - to override each initializer (maybe disabling some irrelevant
initializers, see other item)
 - to check that each resource is properly released in dealloc
 - to check your property attributes for stupid mistakes; add the nonatomic
attribute for properties in classes which are not thread-safe (when omitted
properties are atomic by default); you do not want to pay for some sync you
do not use
 - to create a description function for use with "%@"
 - to implement optiona protocols (remove the now unnecessary stubs)
 - document your class interface (do not forget to mention thread-safety)

 67. (Idea for writing best practice book / document): Make item title as
short and clear as possible. Class items by topic (main book layout), but
also provide other ways to class them (e.g. items before development, during
development, after development, for polishing phase, refactoring, etc.)
 
 
