\chapter{Resource management}
This chapter is mostly devoted to memory management in a reference-counted environment, but also discusses management of scarce resources (file descriptors, buffers, handles to kernel objects, etc.).

Since Mac OS X 10.5, the language runtime supports both manual memory management through reference-counting as well as automatic management using the garbage collector. As for any environment requiring the programmer to manage memory manually, good practices are of great help to avoid leaking memory unnecessarily, and to help the programmer write better code. The main focus of this chapter is therefore on reference-counting. Garbage-collection is only briefly discussed, mostly in the context of writing code compatible with both approaches.

Without good practices, reference-counting in Objective-C can lead to the same problems encountered when using the malloc and free functions in C, or operators new and delete in C++. But thanks to features of Objective-C and conventions applied consistently among the Objective-C programming community, most of these issues can be circumvented.

\section{Understand how reference-counting works in Objective-C}\label{sec:refCountingBasics}
In C++, programmers are used to smart pointers, which are simple objects acquiring ownership of a resource, usually a piece of memory allocated from the heap, and responsible of releasing this resource when the program is done with it. The name smart pointer stems from the fact that, thanks to C++ support for operator overloading, such objects can be given pointer-like semantics. Unlike raw C pointers, smart pointers are not language built-in features, though, but a combination of the C++ powerful concepts of destructor, operator overloading and templates.

Among the several smart pointer flavors encountered in C++, two are of particular interest from the point of view of an Objective-C programmer:
\begin{itemize}
\item Reference-counted pointers, which count how many times the pointer is referenced by the running program. When the counter variable drops to zero, the managed resource is immediately released. Such pointers represent strong references, and suffer from retain cycle issues: If two objects each maintain a strong reference to each other, they will ultimately collapse into an unreachable object pair, leading to a memory leak. The same problem can of course also arise if a chain of object connected by reference-counted pointers closes itself somehow.
\item Weak pointers, which have been introduced to solve the retain cycle problem. Such pointers participate in the reference count, but with the special property that objects only referenced through weak pointers are destroyed. By identifying the head and the tail of a chain of objects connected through strong references, one can therefore break the cycle by having the tail only store a weak reference to the head. In object pairs, this ultimately translates into identifying a parent object which will store the strong reference. One could use raw pointers as weak pointers, but weak pointers objects have still been created to offer more robustness against use of an unreleased ressource.
\end{itemize}

The C / C++ runtime leaves the programer completely free to choose how shes wants to manage memory, whether through manual allocation and deallocation, or by using smart pointer objects. In Objective-C, no C++-like smart pointer objects can exist due to the lack of some of C++ built-in features. Nevertheless, non-GC Objective-C still offers a reference-counted strategy for objects allocated on the heap: Objects are identified by a raw pointer to their memory location, but maintain a reference counter. In the case of C++ reference-counted pointers, the reference counter is transparently increased when copying or destroying pointers, thanks to the use of C++ destructors and operator overloading. Objective-C does not have any of these features and therefore requires the programmer to explicitly ask the object to increment or decrement its counter. When the counter drops to zero, the dealloc message is sent to the object and the associated memory is reclaimed. Objective-C therefore only frees the programmer from defining a private member for the reference counter and from calling dealloc when the object is no longer in use, but still requires a careful manual management of the reference counter.

Since dealloc is called automatically when the reference counter reaches zero, you must never call dealloc directly when you want to free an object. The only case in which you can (and must!) call dealloc is when implementing the dealloc method in class hierarchies: In order for the resources to be released, each dealloc method must namely call its parent counterpart.

In Objective-C, many messages may alter the reference-count of an object. Most of them are directly inherited from the NSObject class. These concepts are therefore not part of the Objective-C runtime but belong to the fundamental framework. This explains why you should always derive your root classes from NSObject. Moreover, you should never have to override those methods, at least not in a way that their expected behavior is altered (you could imagine adding some logging or check features, though):
\begin{itemize}
\item alloc and allocWithZone:: When creating a new object from scratch, its reference count is automatically set to 1
\item retain: You send this message when you need an additional strong reference to an object, effectively adding 1 to the internal counter
\item release: You send this message when you want a strong reference to be immediately released, effectively removing 1 from the internal counter
\item autorelease: In C++, it is safe and considered good practice to return a reference-counted pointer to a resource from a function which has acquired it. Since C++-like smart pointers cannot be effectively implemented in Objective-C, no such idiom can exist in this language. This is why the autorelease message was introduced. By sending an autorelease message to an object, you ask it to register itself with an object pool for some time, and to decrease its reference counter. If the counter reaches zero when the object is still registered with the pool, the object will not be sent the dealloc message, but will stay available until the pool is emptied. During that time, you can still safely use the object or reclaim it again if you need to keep it alive longer. When the pool is cleared, any object whithin it having a retain count of zero is sent the dealloc message. Otherwise it is simply unregistered from the pool, keeping its reference counter intact. As a general rule, an object remains valid within the block it is sent an autorelease message in. This may not be true in cases where autorelease messages and pool management are interleaved, but this is the exception, not the rule. Moreover, objects received from functions are generally autoreleased (see \ref{sec:refCountingConventions}. You can safely assume that such objects can still be used, even if they were sent the autorelease message higher in the function call stack. For more information about autorelease pool management, see \ref{sec:autoreleasePools}.
\end{itemize}
Moreover, messages whose name begins with new, copy or mutableCopy always yield a newly built object with a reference counter of 1.

Note that there is no special C++-like weak pointer concept in Objective-C. When you need a weak reference to an object, simply store the raw object pointer without retaining it.

\subsection{Things to remember}
\begin{itemize}
\item Objective-C offers a reference-counted mechanism for objects allocated on the heap.
\item The reference counter is set to 1 when sending alloc, new, copy or mutableCopy messages.
\item The reference counter is increased when retain is sent.
\item The release and autorelease messages decrease the reference counter. An object which was only sent release messages will be sent dealloc immediately when its reference count drops to zero. An object which received at least an autorelease message will be deallocated some time after its reference count reaches zero.
\item Never call dealloc directly, except for calling the parent class dealloc function when implementing a dealloc method.
\item When a weak reference to an object is needed, simply store the raw pointer object without retaining it.
\end{itemize}

\section{Know and follow the conventions governing memory management}\label{sec:refCountingConventions}
A function or method can return an object it created in two different ways. It can either return the object as is, requiring the caller to release it when it is done with it or, thanks to the autorelease concept, it can return an object waiting for deallocation.

From a caller's perspective, how can you know whether you are responsible of releasing the object you receive or not, i.e. whether you own the returned object or not? You could of course test the retain count of the object just after receiving it (by sending it the retainCount message), but this would be extremely painful and would clutter your code with tests, preventing you from chaining function calls. The solution adopted in Objective-C is to rely on conventions based on the name of the functions or methods you call. Thankfully, the conventions governing objects returned from functions or methods are very easy to remember:
\begin{itemize}
\item alloc / init, allocWithZone / init, as well as the new, copy and mutableCopy methods, return objects you own with a retain count of 1.
\item In all other cases, functions and methods return either autoreleased objects you do not own with a retain count of 0, or static constant objects.
\end{itemize}

These rule are followed consistently across the Objective-C community, it is therefore mandatory that you stick to them as well when writing your own code. These naming conventions are part of a function's interface in Objective-C, and breaking them is the shortest way to disaster. You definitely do not want to use code which does not conform to the rules or to publish code which other programmers cannot trust (you have self esteem, right?)

It is also important that the retain counts returned are always either 0 or 1. If it were not the case, you would have no mean of guessing how many times you must release an object you receive (as said before, asking an object for its retain count is not an option).

So far for naming conventions in interfaces, let us talk about implementation issues now. Fortunately. there is a single rule to follow in order to minimize memory leaks: Each piece of code must have no overall effect on retain counts, except those advertised by its interface. Most notably, this means that:
\begin{itemize}
\item Within a given block of code, each operation which increments a retain count must be compensated by an operation which decrements it. In functions or methods returning an object with a retain count of 1, the only exception is of course the object creation itself.
\item During its lifetime, if an object acquires resources, it must release them accordingly, so that when the object is deallocated the overall effect on retain counts is neutral.
\end{itemize}

A corollary of the previous rules is that you must never release resource in a piece of code which did not acquire it. Overreleasing an object crashes your application and is not easy to debug and fix. Matters are even worse with autoreleased objects, since you application will not crash immediately, but when the pool is emptied.

There is one more case to discuss: Objects which acquire ownership of an object you pass to them. This is for example the case when you are adding objects to an NSMutableArray. Collections of the foundation framework always own the objects they are given. If you ever need to write collections, also try to stick to this rule. In all other cases, there is no naming convention, so be sure to cleanly document your interfaces in the case an object acquires ownership of an object it receives. This is automatically the case when you use properties, since attributes state this behavior explicitly, but in other cases (e.g. init methods) you should consider adding a small snippet of documentation. This way your clients will be confident that they can pass your object a temporary one without having to retain it themselves.

\subsection{Things to remember}
\begin{itemize}
\item Know the naming conventions related to naming conventions, trust them and ensure that your code also fulfills them.
\item alloc / init, allocWithZone / init, as well as new, copy and mutableCopy, return objects you own with a retain count of 1. In all other cases, functions returning objects return autoreleased instances with a retain count of 0 or static constant objects.
\item Each piece of code must have no overall effect on retain counts, except those advertised by its interface.
\item Beware of overreleased objects, which can lead to bugs which are difficult to track and fix.
\item Collections acquire ownership of the objects you add to them. When writing your own collections, follow the same rule.
\item Property attributes are a clean way of stating explicitly whether an object acquires ownership of another one or not.
\item In all cases where naming conventions are not sufficient to guess how a function may alter retain counts, document the interface.
\end{itemize}

\section{Use autorelease pools wisely}\label{sec:autoreleasePools}
An autorelease pool (NSAutoreleasePool) is basically a way of deferring object deallocation(see \ref{sec:refCountingBasics}. This is namely only when a pool is cleaned that autoreleased objects registered with it and having a retain count of zero are sent a dealloc message. Otherwise they simply gets unregistered but are left alive.

Since autorelease and NSAutoreleasePool are not part of the Objective-C runtime but belong to the fundamental framework, pool management is the responsibility of the programmer. You are not limited to the use of one autorelease pool. Each thread namely manages its own pool stack, with the most recently created pool on top of it. But at least you should ensure that a pool is created when a thread starts, and gets destroyed when the thread exits. Otherwise autoreleasing will fail. This of course holds for the main thread as well.\footnote{If you are using Xcode, project templates generate the corresponding code for you.} 

It is important that a function or a class which creates an autorelease pool also releases it when it is done with it. Failure to release a pool at the same level it was created does not leak memory, but should be frowned upon. Consider the following example:
\begin{lstlisting}[frame=single]
- (void)method
{
    // Create a pool to reduce the duration of the MemoryBlock object memory footprint
    NSAutoreleasePool *pool = [[NSAutoreleasePool alloc] init];
    
    MemoryBlock *block = [MemoryBlock memoryBlockWithSizeInMB:256];
    
    // Assume this method creates an autorelease pool but fails to release it
    [self loadFileInMemoryBlock:block];
    
    // This pool is released at the same level it has been created. Good.
    [pool release];
}
\end{lstlisting}
Intuitively we might think that the never released pool will stay forever and that the associated memory will be leaked. This is not the way the pool stack works: When a pool is released, all pools located above it in the stack are namely destroyed. The example above therefore does not leak memory. But hey, that is not the kind of code you want to write, is it?

The use of several autorelease pools per thread provides a finer-grained control of memory usage, especially when many autoreleased objects are created locally.\footnote{Frameworks usually create additional autorelease pools. For example, the iOS UIKit framework creates an autorelease pool when a run loop begins, and releases it when the run loop ends.} When an autorelease message is sent to an object within a given thread, the object is nameley registered with the pool on top of the stack. By enclosing code which locally generates many autoreleased objects within a pool creation / destruction pair, you can reduce the duration of the associated memory footprint.

Even when many temporary objects are created locally, it might not make sense to introduce additional autorelease pools. Consider for example the following code:
\begin{lstlisting}[frame=single]
- (void)method
{
    // Create a pool to collect temporary objects
    NSAutoreleasePool *pool = [[NSAutoreleasePool alloc] init];
    
    // Create lots of heavy temporary objects using the factory method
    for (int i = 1000; i < 10000; ++i) {
        HeavyObject *object = [HeavyObject heavyObjectWithWeight:i];
        // Use object
    }
    
    // Release the pool at the same level it has been created in. Good.
    [pool release];
}
\end{lstlisting}
If all you have is a method returning you temporary objects, or if the objects are quite small, the above solution of course works fine. But here, we can directly create and release the objects within the loop instead of using the factory method. This way there is no need for an autorelease pool anymore:
\begin{lstlisting}[frame=single]
- (void)method
{
    // Create lots of heavy temporary objects
    for (int i = 1000; i < 10000; ++i) {
        HeavyObject *object = [[HeavyObject alloc] initWithWeight:i];
        // Use object
        [object release];
    }
}
\end{lstlisting}
Moreover, a pool is unnecessary if the temporary objects get retained somehow:
\begin{lstlisting}[frame=single]
- (void)method
{
    // Create lots of heavy temporary objects using the factory method
    for (int i = 1000; i < 10000; ++i) {
        HeavyObject *object = [HeavyObject heavyObjectWithWeight:i];
        
        // Mutable array; retain objects which are added to it
        [m_heavyObjects addObject:object];
    }
}
\end{lstlisting}

\subsection{Things to remember}
\begin{itemize}
\item Autorelease pools must be managed by the programmer. An autorelease pool should be created and destroyed at the same level.
\item At least one autorelease pool must be created for each thread. This includes the application main thread.
\item Read the documentation of the frameworks you use to know about how they may use additional memory pools.
\item Do not create autorelease pools if you do not have to. Sometimes autorelease objects can be replaced by objects you can release directly.
\end{itemize}

\section{Use dealloc for releasing memory resources only}
Scarce resources (file descriptors, network connections, etc.) should not be released in dealloc for several reasons:
\begin{itemize}
\item dealloc might not be called due to a bug in reference-counting management or during application tear-down.
\item dealloc might not be called when the last reference is dropped due to a subscription with an autorelease pool
\end{itemize}
To release resources in a predictable way, you should move the corresponding call to a dedicated function. When a scarce resource is not used anymore, simply call this function (which usually should happen just before the resource object is released). This way the scarce resource is guaranteed to be released, even if the object does not get deallocated on the spot, or if it never gets deallocated at all because of some bug.

Moreover, the dealloc function is not used when the garbage-collector is enabled. By moving code which is not related to memory management outside of dealloc, you ensure that your code base can be used with or without garbage-collection, which is especially important for libraries. This also explains why you should only use dealloc for cleaning up memory resources acquired by an object. When garbage-collection is enabled, this mechanism will be completely taken over by the collector.

\subsection{Things to remember}
\begin{itemize}
\item Use dealloc to release memory resources aquired by an object. This ensures that your code is compatible with garbage-collection enabled or not.
\item Move all other cleanup code outside of dealloc, and call it when you know you are done with the object. This is especially important to avoid wasting scarce resources.
\end{itemize}

\section{Use properties to automate memory management}
In C++, smart pointers take care of releasing the resource they owned prior to beassigned a new resource. This very useful mechanism helps avoiding memory leaks and relives the programmer from low-level ownership management. Since smart pointer objects cannot be implemented in Objective-C, this mechanism is not available as is, but can nonetheless be implemented thanks to properties.

Consider how you would naively implenent a Magic 8-ball class, which provides you with a shake method to ask for a new nasty message by shaking it, a rub method to get a new nice message by rubbing it, and a readonly fortuneMessage accessor for reading the current message:
\begin{lstlisting}[frame=single, caption={MagicEightBall.h}]
@interface MagicEightBall {
    NSString *m_fortuneMessage;
}

- (void)shake;
- (void)rub;

@property (readonly) NSString *fortuneMessage

@end
\end{lstlisting}
The implementation is straightforward:
\begin{lstlisting}[frame=single, caption={MagicEightBall.m}]
@implementation MagicEightBall

- (void)dealloc
{
    [m_fortuneMessage release];
    [super dealloc];
}

- (void)shake
{
    [self emitShakeSound];
    [m_fortuneMessage release];
    m_fortuneMessage = [[self pickNastyRandomMessage] copy];
}

- (void)rub
{
    [self emitRubSound];
    [m_fortuneMessage release];
    m_fortuneMessage = [[self pickNiceRandomMessage] copy];
}

@synthesize fortuneMessage = m_fortuneMessage;

@end
\end{lstlisting}
The pickNastyRandomMessage and pickNiceRandomMessage are part of the private class interface, not shown here (TODO: reference to item discussing private interfaces) but, as their name suggest, they return an object we do not own and which I here chose to copy. Before replacing the previous m_fortuneMessage value, we need to release it. Here we have two methods were this value is changed, but imagine if it was changed in several places within the implementation file. If we ever forget to release the old value before replacing it, we create a memory leak. You will probably be careful enough not to make this mistake when you first implement the class, it is very likely that this will not hold as the class evolves.

Fortunately, this problem can easily be solved by using properties. As a preliminary remark, we recall some limitations of properties:
\begin{itemize}
\item Two properties cannot be backed up by the same class member.
\item A property can be overriden in a category or in an extension but, except for the readonly / readwrite attributes, the prototype must be left unaltered.
\item The @synthesize keyword cannot be used with categories, but works fine with extensions.
\end{itemize}
In our case, we can therefore use extensions to declare and synthesize a readwrite version of the fortuneMessage property for use in the implementation file only. This property must have copy semantics to implement the behavior we had:
\begin{lstlisting}[frame=single, caption={MagicEightBall.m}]
@interface MagicEightBall ()
@property (readwrite, copy) NSString *fortuneMessage;
@end

@implementation MagicEightBall

- (void)dealloc
{
    self.fortuneMessage = nil;
    [super dealloc];
}

- (void)shake
{
    [self emitShakeSound];
    self.fortuneMessage = [self pickNastyRandomMessage];
}

- (void)rub
{
    [self emitRubSound];
    self.fortuneMessage = [self pickNiceRandomMessage];
}

@synthesize fortuneMessage = m_fortuneMessage;

@end
\end{lstlisting}
Note that the two versions of the property are generated by the same @synthesize directive, since the main class interface and the extended interface are both implemented by the same @implementation block. Also note that the extension interface has been adeed to the inaccessible .m file in order to keep it private (TODO: reference to private interface section). The readwrite attribute has been here added explicitly but is the default when omitted. We now never update the value of the m_fortuneMessage member, but use the property instead. The property takes care of releasing the old value and of copying the new one. Even dealloc is now implemented by setting the property to nil. Use of the underlying m_fortuneMessage member should in fact now be limited to the implementation of the associated property itself.

The class interface is not altered, except for matching the two property declarations:
\begin{lstlisting}[frame=single, caption={MagicEightBall.h}]
@interface MagicEightBall {
    NSString *m_fortuneMessage;
}

- (void)shake;
- (void)rub;

@property (readonly, copy) NSString *fortuneMessage

@end
\end{lstlisting}
The added copy attribute does not offer more functionality from a client's perspective.




TODO: Then add notification when value changed

TODO: Threading

TODO: Mention easy change of semantics (retain -> copy) in most cases, except of course when a property has not been synthesized

TODO: Always use self.prop, except when implementing without synthesizing

TODO: Sadly maybe still issues with GC: Since properties can be fatter (i.e. do more work than simply releasing memory) we still must set some members to nil prior to a call to release if these members setters not only release memory (e.g. perform notification, release scarce resources, etc.). Except if GC sets to nil by using the setters, in which case everything is consistent!









//*************** Following design is incorrect, but idea - problem is correctly formulated **********

63. Suppose you have a readonly property which reflects some inner status of
an object, e.g. an NSString. Imagine e.g. an object which has some function
renewMessage which, when called, builds a random fortune message
(fortune-teller). You can access the current message using a readonly
property fortuneMessage:

@interface Fortune {
   NSString *m_fortuneMessage;
}
- (void)renewMessage;

@property (readonly) NSString *fortuneMessage;

@end

The last message which was built is stored in some internal member variable
NSString *m_fortuneMessage.

The property is simply synthesized as
@synthesize fortuneMessage = m_fortuneMessage;

But now: how do you write renewMessage? The old value must be discarded each
time a new one is built:

- (void)renewMessage
{
   [m_fortuneMessage release];
   // generateFortuneMessage conforms to Objective-C conventions
   // and returns an autoreleased NSString *
   m_fortuneMessage = [[self generateFortuneMessage] retain];
}

The first time the function is called, fortuneMessage is nil, the release is
a no-op. When later called, the release will take care of releasing the
previous value.

Note that the final release is done in dealloc:

- (void)dealloc
{
   [m_fortuneMessage release];
   [super dealloc];
}

Now imagine a similar class where there are many different ways of setting
fortuneMessage. How can you avoid littering your code with [m_fortuneMessage
release] calls (and minimize bugs due to forgetting one such call)? And how
bad does it get when we want this class to be thread-safe?

Answer: Simply add a fortuneMessageRw (read-write) property to the private
interface of the Fortune class (use categories), with retain attribute (you
might also add the nonatomic attribute if you do not need thread-safety).
This way we do not expose the setter in the public interface:

@interface Fortune (hiddenInterface)
@property (retain) NSString *fortuneMessageRw;
@end

simply synthesized as

@synthesize fortuneMessageRw = m_fortuneMessage;

Now renewMessage can be simply implemented as follows:

- (void)renewMessage
{
   self.fortuneMessageRw = [self generateFortuneMessage];
}

63 (complem.): Usually, not retain for NSString, but copy.

64. Fix for hidden interfaces: Use extensions, not categories (extension =
category without name, e.g. @interface MyClass () @end)

This is because @synthesize does not work with categories, but work with
extensions. It is also nice because the extension methods must be defined in
the same @implementation block as the main class, whereas this is not the
case for categories (whose methods must be implemented in @implementation
MyClass (categoryName) @end )

63 (complem.): Does not work: Two properties cannot be bound to the same
member variable

-> only solution: create private method for setting with copy / release
semantics :-(

63 (complem.) Mmmmh. Maybe possible to declare external interface as
readonly, but to provide a private setter function. Also beware of semantics
(retain then release; check that not receiving the same object when copying;
etc.)

//*************** Works, but not the best method. Worth mentioning it **********
//*************** to pinpoint the fact that there is no need to do that, *******
//*************** properties have a nice feature which solve the ***************
//*************** problem nicely, see below

63. (A better approach, but not optimal (optimal approach follows)) The idea
is to have the getter public and the setter private. We create the setter
manually in the private interface. This way, the setter can be used to
implement the proper setter behavior (here release, then copy). Note that in
the implementation file where the private interface is defined, we can now
use property access (dot operator) to also set the property, thanks to the
naming conventions of Objective-C. But the problem is that the setter must
match other semantics of the public getter (here nonatomic), and this has to
be implemented manually

main.m
-------

#import <Foundation/Foundation.h>

#import "Calendar.h"

int main(int argc, char *argv[])
{
 Calendar *cal = [[Calendar alloc] init];
 NSLog(@"Day 1: %@", cal.day);
 [cal nextDay];
 NSLog(@"Day 2: %@", cal.day);
}

Calendar.h
----------
#import <Foundation/Foundation.h>

@interface Calendar : NSObject {
@private
 NSString *m_day;
}

- (void)nextDay;

@property (nonatomic, readonly) NSString *day;

@end

Calendar.m
----------
#import "Calendar.h"

NSString * c_days[] = {@"Monday", @"Tuesday", @"Wednesday", @"Thursday",
@"Friday", @"Saturday", @"Sunday"};

@interface Calendar ()

- (void)setDay:(NSString *)day;

@end

@implementation Calendar

+ (void)initialize
{
 srand(time(NULL));
}

- (id)init
{
 if (self = [super init]) {
   [self nextDay];
 }
 return self;
}

- (void)nextDay
{
 // Can use usual message passing form
 //[self setDay:c_days[rand() % 7]];

 // or even property call: now the property is writable!
 self.day = c_days[rand() % 7];
}

@synthesize day = m_day;

// Not optimal: We have to match the external property attributes
// manually. Here nonatomic, therefore no @synchronized
- (void)setDay:(NSString *)day
{
 // In general we should test if day is m_day before releasing, otherwise
 // the copy will fail. But here this is not necessary since this can
 // never happen
 [m_day release];
 m_day = [day copy];
}

@end

//*************** end ***************

When examined in isolation, the reference-counting mechanism of Objective-C is no more convenient than the usual malloc / free pair of C. But combined 

63. (THE solution) See "Property re-declaration" in
http://developer.apple.com/mac/library/documentation/cocoa/conceptual/Object
iveC/Articles/ocProperties.html: declare readonly in public interface (with
copy attribute, but the copy is just here so that any redeclaration of the
property matches this attribute, and this is one we will need in the private
redeclaration for the setter). Then redeclare readwrite in private interface
(extension) with same attributes:


main.m
------
#import <Foundation/Foundation.h>

#import "Calendar.h"

int main(int argc, char *argv[])
{
 Calendar *cal = [[Calendar alloc] init];
 NSLog(@"Day 1: %@", cal.day);
 [cal nextDay];
 NSLog(@"Day 2: %@", cal.day);
}


Calendar.h
----------

#import <Foundation/Foundation.h>

@interface Calendar : NSObject {
@private
 NSString *m_day;
}

- (void)nextDay;

// here readonly (readonly & readwrite are the only attributes allowed to
// mismatch between property declaration and re-declarations). The copy
// attribute is irrelevant for the public interface but is required so
// that re-declarations match
@property (nonatomic, readonly, copy) NSString *day;

@end


Calendar.m
----------
#import "Calendar.h"

NSString * c_days[] = {@"Monday", @"Tuesday", @"Wednesday", @"Thursday",
                      @"Friday", @"Saturday", @"Sunday"};

@interface Calendar ()

- (void)setDay:(NSString *)day;

// here readwrite with copy semantics
@property (nonatomic, readwrite, copy) NSString *day;

@end

@implementation Calendar

+ (void)initialize
{
 srand(time(NULL));
}

- (id)init
{
 if (self = [super init]) {
   [self nextDay];
 }
 return self;
}

- (void)nextDay
{
 self.day = c_days[rand() % 7];
}

// generate both the public getter and the private setter with copy
// semantics
@synthesize day = m_day;

@end

63. (Remark) Of course, even if no public getter is defined, having a
private setter can eliminate the need for manual retain / release management
littering code.
(Remark2): The readwrite attribute is the default one, you do not need to
use it in the private property re-definition. Just omit the readonly.

 65 (not perfect, 65b is far better). Good practice for 63: In C++, const
variables are assigned in an initializer list once for all. There is no
support for such a concept in Objective-C, but still some variables are set
once at the object creation. For such variables, having a private property
is in general useless. For any other internal member which might change
during the object lifetime, consider providing a private setter (at least; a
public readonly accessor might still be useful). This way you can forget
about memory management for such objects, and when you are done with them or
want to reset them, you just have to assign them nil. In dealloc, instead of
calling release on objects having a property, simply assign nil.

A good idea: Introduce a notation convention for such "final" members, e.g.
mf_variableName.
The term "final" is best suited, because such members are not necessarily
const. Understand final here as "set once during object creation, and then
never replaced (but maybe changed)". Imagine e.g. a mutable array whose
purpose is to contain the results within a datasource object. The datasource
object can be sent different requests, and before each request the array
must be flushed to contain the new results. Such an array is final.
On the contrary, if the datasource were to store its results in an immutable
array, the corresponding member would not be final since it must be replaced
each time a new request is performed.

Remark: Private properties are meaningful for final objects. final instance
variables of primitive type do not suffer from memory management issues and
do not benefit (in general) from having such a private property, except when
its purpose is to encapsulate some other piece of work when setting /
getting a value. In all other cases, do not define a private property.

Remark 2: Prefixes in notation for class members always begin with m, then
add the other modifiers. The reason is that this allows to find all members
use auto-completion by entering only m and pressing ESC. In general, when
designing prefixes, find which ones design subsets and which ones supersets,
and order the prefixes accordingly.

The problem: Writing the dealloc is tedious since some objects have a
property (private setter or public) and other not. This is difficult to
maintain. Knowing that the objects are final do not convey much information
in Objective-C and could be a pain in the ass if some time later the final
character is dropped.

65b. (Excellent) No initializing concept in Objective-C and no support,
therefore do NOT introduce one artificially. Just apply the following rule:
All member objects must have a corresponding setter (at least private). In
the implementation of a class, *always* use those setters (except in the
setter / getter implementation of course!), *never* access member variables
directly. In the destructor, set all objects to nil.
This way, encapsulation is preserved, the correct behavior is applied when
getting / setting values, and additional instructions can be bundled with
those operations consistently. Moreover, setting to nil is by construction
equivalent to calling release. This way the code is similar to the one
written with garbage collection enabled on MacOS! Moreover this is much
easier to follow than 65, and it is easier to maintain consistently. The
only issue is performance, but Objective-C is already a "slow" language.
Moreover, if performance is at premium, some classes can be written without
this convention where needed. As always, flexibility and encapsulation is
traded against performance. But this should not really be an issue 90% of
the time. Of course, this requires writing some more code (but only a little
more thanks to property synthesization), but at least this also forces the
programmer to carefully think about threading issues when accessing objects.

Remark: For primitive types, only declare private properties if you need to
bundle some code with a set / get operation. Primitive types do not suffer
from memory management issues, therefore "bare" properties would bring
nothing more. Of course you still might want to provide a public properties
for client to access an internal variable.

Remark 2: Using this approach, you will usually assign autorelease-d objects
directly:
 self.someMember = [[[Clazz alloc] init] autorelease];
Another idiom when you cannot autorelease (e.g. when creating a crazy number
of objects in a loop) is:
 self.someMember = [[Clazz alloc] init];
 [self.someMember release];

Remark 3: Private properties also motivate why it is not wise to create a
property which returns a type different from the one of the underlying
member (most probably upcasted). See 69.

66. When implementing a class, access member variables through their
properties as much as possible. This also includes the dealloc function in
which you should call self.variable = nil. The reason is the following: your
setter was maybe performing some additional work for you, it might therefore
also was implemented to perform some cleanup work when nil was assigned.
Simply releasing without the setter would have required you to duplicate
this code in the dealloc function.
Always using setters therefore increases encapsulation and provides the
correct behavior when assigning and dealloc-ing.

65b (complem.): Do not write private property for members of primitive type,
except if:
- some code must be bundled with the set, the get, or both
- a public property exists for getting the value; by defining a private
property for the set, you ensure that the variable can be accessed
consistently within the implementation file (otherwise it would be confusing
to access the variable using self.m_name; sometimes this could be dropped,
and if the getter is bundled with code as well, then encapsulation has been
breached).

65b (complem.): By using properties this way, we in fact are close to
the use of smart pointers in C++. No need to manually release the
memory, assignment does it for us.

Warning: GC does not call dealloc anymore! => see scarce resources: dealloc must only be used for memory release



